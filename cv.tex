\documentclass[contbibnum,titleabove]{simplecv}
\usepackage{geometry}
\usepackage{hyperref}
\usepackage{titlesec}
\usepackage{enumitem}
\geometry{
	left=15mm,
	right=15mm,
	top=15mm,
	bottom=15mm
}
\def\typeface{phv}
\headerfont{\fontfamily{\typeface}\selectfont}
\titlefont{\fontfamily{\typeface}\selectfont\Huge}
\sectionfont{\bf\fontfamily{\typeface}\selectfont\large}
\subsectionfont{\bf\fontfamily{\typeface}\selectfont}
\topiclabelfont{\it\fontfamily{\typeface}\selectfont}
\topictitlefont{\fontfamily{\typeface}\selectfont}
\renewcommand{\topicmargin}{8em}
\setlength{\parindent}{0cm}
\titlespacing\section{0pt}{8pt plus 2pt minus 2pt}{8pt plus 2pt minus 2pt}
\titlespacing\subsection{0pt}{8pt plus 2pt minus 2pt}{8pt plus 2pt minus 2pt}
\setlist{noitemsep}
\setenumerate{noitemsep}

\leftheader{5301, James Clerk Maxwell Building\\Peter Guthrie Tait Road\\Edinburgh, EH9 3FD}
\rightheader[r]{+44 7719 773067\\{\tt adam.morris@cern.ch}}
\title{Adam Morris}
\newcommand\dateditem[2]{#1\hfill#2\\}
\newcommand\topictitle[3]{\dateditem{{\textbf{#1}}}{#3}#2}
\begin{document}
	\maketitle
	\fontfamily{\typeface}\selectfont
%	\section{Personal statement}
	I am a finishing PhD student and short-term postdoc at the University of Edinburgh LHCb group.
	I will be able to start a new job in January 2018.
	My research experience is in studies of four-body charmless hadronic decays of $B$ mesons.
	My thesis work comprises a measurement of the $B^0_s \to \phi \phi$ branching fraction, with a search for $B^0 \to \phi \phi$, and an amplitude analysis of $B^0_s \to \phi K^{+}K^{-}$ decays.
	I have also worked on RICH detector operations, monitoring and studies for the upgrade.
	My research interests lie in experimental flavour physics, particularly precision measurements of rare processes and exotic hadron spectroscopy.
	\section{Employment}
	\topictitle{Postdoctoral research associate}{University of Edinburgh}{Aug 2017 --- Dec 2017}
	\begin{topic}
		\topsep0em
		\item[Description]{Four month contract to work on an amplitude analysis of $\Lambda_b \to \chi_{c\{1,2\}} p K^{-}$ decays.}
	\end{topic}
	\section{Education}
	\topictitle{PhD in experimental particle physics}{University of Edinburgh}{Sep 2013 --- Sep 2017}
	\begin{topic}
	\itemsep-0.3em
		\item[Thesis title]{\textit{Measurements of charmless $B_s$ meson decays at LHCb}}
		\item[Supervisors]{Matthew Needham and Franz Muheim}
		\item[Status]{Thesis submitted; awaiting Viva}
	\end{topic}
	\topictitle{MSci in physics}{University of Birmingham}{Oct 2009 --- Jul 2013}
	\begin{topic}
	\itemsep-0.3em
		\item[Thesis title]{\textit{A possible lepton flavour violation search at NA62}}
		\item[Supervisors]{Cristina Lazzeroni and Evgueni Goudzovski}
		\item[Classification]{Upper second class honours}
	\end{topic}
	\section{Physics analysis}
	\topictitle{Amplitude analysis of $\Lambda_b \to \chi_{c\{1,2\}} p K^{-}$ decays}{}{2017 ---}
	Amplitude fits to $\Lambda_b \to \chi_{c1} p K^{-}$ and $\Lambda_b \to \chi_{c2} p K^{-}$ decays using Run 1 and 2 data, with a focus on searching for pentaquarks decaying to $\chi_{c\{1,2\}} p$.
	My role is to develop the eight-dimensional fitter, within a framework based on TensorFlow, to run on a GPU cluster at Edinburgh.
	\\
	\topictitle{Amplitude analysis of $B^0_s \to \phi K^{+}K^{-}$ decays}{}{2015 --- 2017}
	An amplitude fit to ${B^0_s \to \phi K^{+}K^{-}}$ events with ${m(K^{+}K^{-})<1.8}$~GeV. The decay ${B^0_s \to \phi f_2^\prime(1525)}$ is observed for the first time, and measurements are made of its branching fraction and longitudinal polarisation fraction.
	This is my primary thesis work, and I was responsible for performing the analysis, from selection, modelling the acceptance, signal and background, fitting the data and calculating results and systematic uncertainties.
	\\
	\topictitle{Measurement of the $B^0_s \to \phi\phi$ branching fraction and search for $B^0 \to \phi \phi$~\textnormal{\cite{phiphiBF}}}{}{2014 --- 2015}
	The $B^0_s \to \phi\phi$ branching fraction was measured using $B^0 \to \phi K^{*0}$ for normalisation, resulting in a factor of five reduction in the statistical uncertainty compared to the previous best result.
%	This is an important normalisation channel for charmless $B$ decay measurements, such as $B_s \to \phi \pi^{+} \pi^{-}$.
	In addition, a search was performed for the suppressed $B^0 \to \phi\phi$ decay, and an upper limit placed on its branching fraction, which is a factor of seven improvement compared to the previous best result.
	This analysis forms part of my thesis work, and I was responsible for the mass fits and most of the calculations of efficiencies and systematics, as well as setting the limit on the $B^0$ mode using the $CL_s$ method.
	\\
	\topictitle{Measurement of CP violation in $B^0_s \to \phi\phi$ decays~\textnormal{\cite{phiphiCPV}}}{}{2013 --- 2014}
	A time-dependent angular fit performed to $B^0_s \to \phi\phi$ events to extract the weak phase $\phi_s$, which is expected to be close to zero due to cancellation between the leading-order mixing and decay diagrams.
	I was responsible for parametrising the peaking backgrounds and performing fits to the $\phi\phi$ invariant mass to obtain event yields.
\newpage
	\section{Hardware and operations}
	\topictitle{LHCb control room shifts (4$\times$ shift leader \& 11$\times$ data manager)}{}{2015 --- 2016}
	\topictitle{RICH piquet shifts (3 weeks)}{}{2015 --- 2016}
	\topictitle{Ion feedback monitoring of the RICH HPDs}{}{2015 --- 2016}
	\topictitle{RICH upgrade testbeam}{}{2014 --- 2016}
	I participated in 4 testbeam periods, during which I took data-taking shifts and improved the automation of the data acquisition and processing. I performed the dark count analysis (see Ref.~\cite{testbeam}) and developed an online alignment tool.
	\topictitle{Performance of MaPMTs in magnetic fields for the RICH upgrade~\cite{mapmt}}{}{2013 --- 2014}
	I measured the drop in efficiency of the multi-anode photomultipliers, selected for the RICH upgrade, as a function of magnetic field strength applied along different axes for different operating voltages.
	This study led to the conclusion that, under the operating conditions at LHCb, magnetic shielding is required to achieve 90\% efficiency for all anodes.
	\section{Schools, workshops and conferences}
	\topictitle{Exotic hadron spectroscopy workshop}{}{University of Edinburgh, September 2016}
	\topictitle{Heavy Quarks and Leptons (HQL) conference}{Talk: \textit{Charmless $b$-meson and $b$-baryon decays at LHCb}~\cite{HQL}}{Virginia Tech, May 2016}\\
	\topictitle{LHCC Poster Session}{Poster: \textit{Measurement of the $B_s \to \phi \phi$ branching fraction and angular analysis of $B_s \to \phi \pi^{+} \pi^{-}$ at LHCb}}{CERN, March 2016}\\
	\topictitle{UK HEP forum: ``Anomalies and Deviations'' workshop}{Poster: \textit{Measurement of the $B_s \to \phi \phi$ branching fraction and search for $B^0 \to \phi \phi$ at LHCb}}{STFC, Abingdon, November 2015}\\
	\topictitle{European School of High-Energy Physics}{}{Bansko, Bulgaria, September 2015}
	\topictitle{Young Experimentalists and Theorists Institute (YETI) workshop}{}{Durham University, January 2015}
	\topictitle{STFC HEP summer school}{}{University of Warwick, September 2014}
	\topictitle{B-physics at frontier machines (Beauty) conference}{Poster: \textit{Measurement of CP violation in $B^0_s \to \phi\phi$ decays~\cite{beauty}}}{University of Edinburgh, July 2014}
	\section{Public outreach}
	\dateditem{Participant in an \href{http://cds.cern.ch/record/2024978}{`Ask Me Anything' session organised by CERN on \textit{reddit.com}}}{May 2015}
	\dateditem{Participant in a \textit{Speed Science} event at the \textit{Glasgow Science Festival}}{June 2014}
	\dateditem{Demonstrator at the \textit{Discovering the Higgs Boson} exhibit at the \textit{Big Bang Fair}, Birmingham NEC}{March 2014}
	\section{Skills}
	\begin{itemize}
		\item HEP data analysis techniques: multivariate selection, maximum-likelihood fits, amplitude analysis and limit-setting
		\item Proficiency with C++, Python and shell scripting.
		\item Familiarity with ROOT (incl. TMVA, RooFit and RooStats), GSL, Boost and the LHCb software
		\item Experience using parallel computing resources, such as the Edinburgh and CERN batch services and the LHC Grid
		\item Frequent user of version control systems (Git and SVN) and experience with collaborative software development
	\end{itemize}
	\section{Publications}
	\renewcommand{\topicmargin}{5em}
	\subsection{Papers}
	\begin{thebibliography}{1}
		\itemsep-0.3em
		\bibitem{phiphiBF}\textit{Measurement of the $B^0_s \to \phi\phi$ branching fraction and search for the decay $B^0 \to \phi \phi$},\\LHCb collaboration, \href{http://dx.doi.org/10.1007/JHEP10(2015)053}{JHEP \textbf{10} (2015) 053}
		\bibitem{phiphiCPV}\textit{Measurement of CP violation in $B^0_s \to \phi\phi$ decays},\\LHCb collaboration, \href{http://dx.doi.org/10.1103/PhysRevD.90.052011}{Phys. Rev. \textbf{D90}, (2014) 052011}
		\bibitem{testbeam}\textit{Test of the photon detection system for the LHCb RICH Upgrade in a charged particle beam},\\M. K. Baszscyk \textit{et al}, \href{http://cds.cern.ch/record/2197586}{LHCb-PUB-2016-019}, to be submitted to JINST
	\end{thebibliography}
	\subsection{Proceedings}
	\begin{thebibliography}{2}
		\itemsep-0.3em
		\bibitem{mapmt}\textit{Characterisation and magnetic field properties of multianode photomultiplier tubes},\\S. Eisenhart \textit{et al}, on behalf of the LHCb RICH collaboration, \href{http://dx.doi.org/10.1016/j.nima.2014.05.036}{Nucl. Instrum. Meth. \textbf{A766} (2014) 167-170}
		\bibitem{HQL}\textit{Charmless b-meson and b-baryon decays at LHCb},\\A. Morris, on behalf of the LHCb collaboration, \href{https://pos.sissa.it/cgi-bin/reader/contribution.cgi?id=274/059}{PoS (HQL 2016) 059}
		\bibitem{beauty}\textit{Measurement of CP violation in $B^0_s \to \phi\phi$ decays},\\A. Morris, on behalf of the LHCb collaboration, \href{https://pos.sissa.it/cgi-bin/reader/contribution.cgi?id=216/064}{PoS (Beauty 2014) 064}
	\end{thebibliography}
\end{document}

